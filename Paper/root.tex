\documentclass[conference]{IEEEtran}



\usepackage{amsmath}
\usepackage{amssymb}
\usepackage{algorithmic}
%\usepackage{url}

% correct bad hyphenation here
\hyphenation{op-tical net-works semi-conduc-tor}


\begin{document}

\title{Variable discount factor learning in Markov Decision Process}


% author names and affiliations
% use a multiple column layout for up to three different
% affiliations
\author{\IEEEauthorblockN{Davide Tateo, Alessandro Nuara, Carlo D'Eramo}
\IEEEauthorblockA{Dipartimento di Elettronica, Informazione e Biongegneria
Politecnico Di Milano\\
Milano, Italy \\
Email: \{davide.tateo, alessandro.nuara, carlo.deramo\}@polimi.it}
}


% make the title area
\maketitle

% As a general rule, do not put math, special symbols or citations
% in the abstract
\begin{abstract}
TODO
\end{abstract}

% no keywords


% For peer review papers, you can put extra information on the cover
% page as needed:
% \ifCLASSOPTIONpeerreview
% \begin{center} \bfseries EDICS Category: 3-BBND \end{center}
% \fi
%
% For peerreview papers, this IEEEtran command inserts a page break and
% creates the second title. It will be ignored for other modes.
\IEEEpeerreviewmaketitle



\section{Introduction}
Motivations. \\

State of the art (Q-Learning~\cite{watkins1992q}, SARSA, Double, Weighted, R-Learning)


\section{Preliminaries}

\section{The proposed Method}

\subsection{Decomposition of the TD error}

Decompose Q function:

\begin{align}
Q(x,u) & =\mathbb{E}\left[R(x,u,x')+\gamma Q(x',\pi(x'))\right] \nonumber\\ 
 & =\mathbb{E}\left[R(x,u,x')\right]+\gamma\mathbb{E}\left[Q(x',\pi(x'))\right] \nonumber\\
 & =\tilde{R}(x,u)+\gamma\tilde{Q}(x,u)
\end{align}

Decomposed TD update:

\begin{align}
\tilde{R}(x,u) & \leftarrow\tilde{R}(x,u)+\alpha(R(x,u,x')-\tilde{R}(x,u))\\
\tilde{Q}(x,u) & \leftarrow\tilde{Q}(x,u)+\beta(Q(x',\pi(x'))-\tilde{Q}(x,u))
\end{align}

Update of the Q function:

\begin{align}
Q(x,u) & \leftarrow\tilde{R}(x,u)+\alpha(R(x,u,x')-\tilde{R}(x,u)) \nonumber\\
 & +\gamma\left(\tilde{Q}(x,u)+\beta(Q(x',\pi(x'))-\tilde{Q}(x,u))\right) \nonumber\\
 & =Q(x,u)+\alpha(R(x,u,x')-\tilde{R}(x,u)) \nonumber\\
 & +\gamma\beta(Q(x',\pi(x'))-\tilde{Q}(x,u))
\end{align}

\subsection{Analysis of the decomposed update}
If $\alpha=\beta$

\begin{align}
Q(x,u) & \leftarrow Q(x,u)+\alpha(R(x,u,x')+\gamma Q(x',\pi(x')) \nonumber\\
 & -Q(x,u))
\end{align}

That is the classical Q-Learning update

If $\beta=\delta\alpha$
\begin{align}
Q(x,u) & \leftarrow Q(x,u)+\alpha(R(x,u,x')+\gamma\delta Q(x',\pi(x')) \nonumber\\
 & -(\tilde{R}(x,u)+\gamma\delta\tilde{Q}(x,u))) \nonumber\\
 & =Q(x,u)+\alpha(R(x,u,x')+\gamma'Q(x',\pi(x')) \nonumber\\
 & -(\tilde{R}(x,u)+\gamma'\tilde{Q}(x,u))) \nonumber\\
 & =Q(x,u)+\alpha((R(x,u,x')+\gamma'Q(x',\pi(x'))) \nonumber\\
 & -Q'(x,u))
\end{align}
With $\gamma'=\gamma\delta$. Notiche that $Q'(x,u)$ is the current Q function with a different learning rate.

\subsection{Variance dependent learning rate}
\begin{align}
 \alpha & =\dfrac{\sigma^{2}}{\sigma^{2}+1}
\end{align}


\section{Experimental results}



\section{Conclusion}


% trigger a \newpage just before the given reference
% number - used to balance the columns on the last page
% adjust value as needed - may need to be readjusted if
% the document is modified later
%\IEEEtriggeratref{8}
% The "triggered" command can be changed if desired:
%\IEEEtriggercmd{\enlargethispage{-5in}}

% references section

% can use a bibliography generated by BibTeX as a .bbl file
% BibTeX documentation can be easily obtained at:
% http://mirror.ctan.org/biblio/bibtex/contrib/doc/
% The IEEEtran BibTeX style support page is at:
% http://www.michaelshell.org/tex/ieeetran/bibtex/
\bibliographystyle{IEEEtran}
% argument is your BibTeX string definitions and bibliography database(s)
\bibliography{biblio}



% that's all folks
\end{document}


